\documentclass[UTF8]{ctexart}

\title{
\textbf{浙江大学 }20 \underline{ 19 } — 20 \underline{ 20 } \textbf{春夏学期} \\
\textbf{《微分几何》课程期末考试试卷}
}

\author{
课程号: \underline{ \quad 06121530 \quad }, 开课学院: \underline{ \quad 数学科学学院 \quad } \\
考试试卷: \checkmark A 卷、 B 卷 (请在选定项上打\checkmark) \\
考试形式: \checkmark 闭、开卷 (请在选定项上打\checkmark), 允许带\underline{ \quad 无 \quad  }进场 \\
考试日期: \underline{ \quad 2020 \quad } 年 \underline{ \quad 09 \quad } 月 \underline{ \quad 02 \quad } 日, 考试时间: \underline{ \quad 120 \quad }分钟 \\
\textbf{诚信考试, 沉着应考, 杜绝违纪}
}

\date{}

\usepackage{geometry}
\usepackage{amsmath}
\usepackage{amsfonts}
\usepackage{amssymb}
\usepackage{array}
\usepackage{graphicx}
\usepackage{subfigure}
\usepackage{enumerate}
\pagestyle{plain}
\geometry{top=2.5cm}
\linespread{1.75}
\renewcommand{\d}{\text{d}}

\begin{document}
\maketitle

\begin{center}
考生姓名: \underline{\quad\quad\quad\quad\quad\quad\quad\quad\quad}  学号: \underline{\quad\quad\quad\quad\quad\quad\quad\quad\quad}  所属院系: \underline{\quad\quad\quad\quad\quad\quad\quad\quad\quad}
\end{center}

\centerline{\Large{\textbf{由 CC98 @Serapay 回忆整理}}}

1. (10分) 已知曲线的方程 $\textbf{x}(t)=(at,bt^2,t^3),-\infty<t<\infty$, 其中 $a,b$ 为常数.
\begin{enumerate}[(1)]
    \item 求曲线的曲率和挠率;
    \item 求曲线为一般螺线的充要条件.
\end{enumerate}

2. (15分) 已知伪球面的方程 $\textbf{x}(u,v)=(a\cos v\cos u,a\cos v\sin u,a\ln(\sec v+\tan v)-a\sin v)$,
$0\le v<\displaystyle\frac{\pi}{2},0\le u\le 2\pi$, $a>0$ 为常数.证明伪球面上所有纬线的测地曲率为 $\displaystyle\frac{1}{a}$,
并利用 Gauss-Bonnet 公式证明伪球面的面积为 $2\pi a^2$.
\\

3. (15分) 证明悬链面 $\textbf{x}(t,\theta)=(a\cosh t\cos\theta,a\cosh t\sin\theta,at),-\infty<t<\infty,0\le\theta\le2\pi$ 和
正螺面 $\textbf{r}(u,v)=(v\cos u,v\sin u,au),0\le u\le2\pi,-\infty<v<\infty$ 之间存在等距对应, 其中 $a$ 为常数.
\\

4. (15分) 求证: Gauss 曲率和平均曲率为常数的曲面为平面、球面或者圆柱面.
\\

5. (15分) 对任意曲面变分, 求证曲面 $M$ 的面积达到逗留值的充要条件是 $H=0$.
\\

6. (15分) 设曲线的平移曲面 $\textbf{x}(u,v)=\textbf{a}(u)+\textbf{b}(v)$, 其中 $u,v$ 分别是曲线的弧长参数. 若参数曲线构成正交网, 证明该曲面为柱面.
\\

7. (15分) 设 $H$ 是空间简单闭曲线 $\Gamma$ 的管状曲面 $M^2$ 的平均曲率, 求证:
\[
\int_{M^2}H^2\d A\ge2\pi^2,
\]
等号成立当且仅当 $\Gamma$ 为圆周, $M^2$ 为圆环面且半径比为 $\displaystyle\frac{1}{\sqrt{2}}$. (课本 P114 定理 5.6)

\end{document} 