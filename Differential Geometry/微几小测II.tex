\documentclass[UTF8]{ctexart}

\title{\LARGE \textbf{浙江大学 $2019-2020$ 学年春夏学期}}
\author{求是数学班《微分几何》 测验 II}
\date{2020.04.10}

\usepackage{geometry}
\usepackage{amsmath}
\usepackage{amsfonts}
\usepackage{amssymb}
\usepackage{array}
\usepackage{graphicx}
\usepackage{subfigure}
\usepackage{enumerate}
\usepackage{bm}
\renewcommand{\d}{\text{d}}

\linespread{1.6}
\pagestyle{plain}
%\geometry{left=2.0cm,top=2.5cm,bottom=2.5cm,right=2.0cm}
\allowdisplaybreaks[4]

\begin{document}
\maketitle
1. (15分) 证明: 曲面为球面 (的一部分) 的充要条件是, 曲面上每一点处的法线都通过一个定点.
\\

2. (20分) (1) 求曲面 $z=f(x,y)$ 的 Gauss 曲率和平均曲率;

(2) 证明: 曲面为球面或平面的充要条件是 $H^2=K$.
\\

3. (15分) 写出环面 $\textbf{r}(u,v)=\left((a+r\cos u)\cos u,(a+r\cos u)\sin v,r\sin u\right)$ 
的 Gauss 公式和 Weingarten 公式, 这里 $a,r$ 是正常数, $a>r$ 且 $-\infty<u<+\infty,-\infty<v<+\infty$.
\\

4. (20分) (1) 若 $z=f(x)+g(y)$ 是极小曲面, 证明: 除相差一个常数外, 它可以写成: $\displaystyle az=\ln\frac{\cos ay}{\cos ax}$;

(2) 求 Gauss 曲率为零的旋转曲面.
\\

5. (15分) 证明: 如果从曲面 $M^2$ 到单位球面 $S^2(1)$ 的 Gauss 映射为共形对应, 则该曲面一定是球面或极小曲面.
\\

6. (15分) 证明: 曲面 $\textbf{x}=\textbf{x}(u^1,u^2)$ 上曲线 $u^1=u^1(s),u^2=u^2(s)$
的测地曲率公式为 $\displaystyle k_g=\left|\left(\textbf{n},\frac{\d\textbf{x}}{\d s},\frac{\d^2\textbf{x}}{\d s^2}\right)\right|$,
其中 $\textbf{n}$ 为曲面的单位法向. 并计算旋转面 $\left(f(u^1)\cos u^2,f(u^1)\sin u^2,u^1\right)$
的纬线和经线的测地曲率.

\end{document}
