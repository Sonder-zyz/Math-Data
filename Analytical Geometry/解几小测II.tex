\documentclass[UTF8]{ctexart}

\title{\textbf{浙江大学 $2017-2018$ 学年秋冬学期}}
\author{求是数学班《几何学》测验 II}
\date{2017.11}

\usepackage{geometry}
\usepackage{amsmath}
\usepackage{amsfonts}
\usepackage{amssymb}
\usepackage{array}
\usepackage{graphicx}
\usepackage{subfigure}
\usepackage{enumerate}

\linespread{1.75}
\pagestyle{plain}
\geometry{left=2.0cm,top=2.5cm,bottom=2.5cm,right=2.0cm}
\allowdisplaybreaks[4]

\begin{document}
\maketitle
1. 以空间直角坐标系的原点为球心, 半径为 $r$ 的球面记为 $S^2(r)$. 过该球面的北极 $N(0,0,r)$ 向 $xOy$ 平面作球极投影 $f:S^2(r)\rightarrow \mathbb{R}^2$, 求该球极投影的表达式.

2. (1) (7分) 判断点 $M(1,1,1)$ 在平面 $7x+2y+z=0$ 与平面 $15x+8y-z-2=0$ 所构成的锐二面角中还是钝二面角中, 并求钝二面角的角平分面的方程.

(2) (8分) 求通过点 $A(4,0,-1)$ 且同时与两直线 
\[
L_1:\frac{x-1}{2}=\frac{y+3}{4}=\frac{z-5}{5},L_2:\frac{x}{5}=\frac{y-2}{-1}=\frac{z+1}{2}
\]
都相交的直线方程.

3. (15分) 设有两条异面直线
\[
L_1:\frac{x-6}{1}=\frac{y}{2}=\frac{z-1}{2},L_2:\frac{x}{1}=\frac{y-8}{2}=\frac{z+4}{-2}.
\]

(1) 求与 $L_1,L_2$ 等距离的点的轨迹;

(2) 设 $\pi_1,\pi_2$ 分别是过直线 $L_1,L_2$ 的平面, 且两平面垂直. 求平面 $\pi_1$ 与 $\pi_2$ 的交线的轨迹方程, 并指出它是什么曲面.

4. (15分) 求所有与直线
\[
L_1:\frac{x-6}{1}=\frac{y}{2}=\frac{z-1}{2},L_2:\frac{x}{1}=\frac{y-8}{2}=\frac{z+4}{-2}
\]
都相交, 且平行于平面 $2x+3y-5=0$ 的直线所构成的图形 $S$ 的方程.

5. (15分) 求单叶双曲面 $\displaystyle\frac{x^2}{a^2}+\frac{y^2}{b^2}-\frac{z^2}{c^2}=1$ 上互相垂直的直母线的交点的轨迹.

6. (15分) 在二次曲面 $2x^2+y^2-z^2+3xy+xz-6z=0$ 上, 求过点 $(1,-4,1)$ 的所有直母线的方程.

7. (15分) 设有二次曲面族 $\displaystyle\frac{x^2}{a^-\lambda}+\frac{y^2}{b^2-\lambda}+\frac{z^2}{c^2-\lambda}=1$ (常数$a>b>c>0$), 对于不等于 $a^2,b^2,c^2$ 的 $\lambda$, 它表示一张二次曲面. 求证: 对于空间内除原点外任意一点 $(x_0,y_0,z_0)$, 恰有二次曲面族中的三张曲面通过, 且它们分别是单叶双曲面、双叶双曲面和椭球面.
\end{document}