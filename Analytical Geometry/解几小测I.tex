\documentclass[UTF8]{ctexart}

\title{\textbf{浙江大学 $2017-2018$ 学年秋冬学期}}
\author{求是数学班《几何学》测验 I}
\date{2017.10}

\usepackage{geometry}
\usepackage{amsmath}
\usepackage{amsfonts}
\usepackage{amssymb}
\usepackage{array}
\usepackage{graphicx}
\usepackage{subfigure}
\usepackage{enumerate}

\linespread{1.75}
\pagestyle{plain}
\geometry{left=2.0cm,top=2.5cm,bottom=2.5cm,right=2.0cm}
\allowdisplaybreaks[4]

\begin{document}
\maketitle

1. (10分) 设 $a,b,c$ 分别是 $\triangle ABC$ 三边 $BC,CA,AB$ 的边长, $\Delta$ 是三角形的面积. 求证: $a^2+b^2+c^2\geq4\Delta$.

2. (10分) 利用向量的方法证明三角形三角形三条中线长度的平方和等于它的三边长度的平方和的 $\displaystyle\frac{3}{4}$.

3. (10分) 已知三面角 $O-ABC$ 中, $\angle BOC=\alpha,\angle COA=\beta,\angle AOB=\gamma$. 利用向量的方法, 求平面 $AOB$ 与 $BOC$ 构成的二面角的平面角.

4. (15分) 已知空间四点 $A,B,C,D$, 其中无三点共线. 求证: 这四点共面的充要条件是存在四个不全为零的实数 $p,q,r,s$ 满足:
\[
p\overrightarrow{OA}+q\overrightarrow{OB}+r\overrightarrow{OC}+s\overrightarrow{OD}=\overrightarrow{0}, p+q+r+s=0.
\]

5. (15分) 设 $A,B,C$ 是空间中三个互不相同的点, 向量 $\overrightarrow{r}=\overrightarrow{OA}\times\overrightarrow{OB}+\overrightarrow{OB}\times\overrightarrow{OC}+\overrightarrow{OC}\times\overrightarrow{OA}$, 求证:

(1) 当 $\overrightarrow{r}=\overrightarrow{0}$ 时, $A,B,C$ 三点共线;

(2) 当 $\overrightarrow{r}\neq\overrightarrow{0}$ 时, $\overrightarrow{r}$ 垂直于 $A,B,C$ 三点确定的平面.

6. (15分) 设 $P$ 是球内一点, 球面上有 $A,B,C$ 三个动点, 使得 $\angle BPA=\angle CPA=\angle CPB=90^{\circ}$. 以 $PA,PB,PC$ 为棱作一长方体, $Q$ 是长方体与 $P$ 相对的顶点. 当 $A,B,C$ 在球面上移动时, 求 $Q$ 点的轨迹.

7. (10分) 已知在 $\triangle ABC$ 中, $D,E,F$ 依次是边 $BC,CA,AB$ 的内点. 求 $\overrightarrow{AD}+\overrightarrow{BE}+\overrightarrow{CF}=\overrightarrow{0}$ 的充要条件 (用点 $D,E,F$ 的位置关系来表示).

8. (15分) 设 $P$ 是平面上以 $O$ 为圆心的单位圆周上的任意一点, $A_1,A_2,\cdots,A_n$ 是单位圆的内接正 $n$ 边形的顶点, 求证:

(1) $\overrightarrow{OA_1}+\overrightarrow{OA_2}+\cdots+\overrightarrow{OA_n}=\overrightarrow{0}$;

(2) $\overrightarrow{PA_1}\cdot\overrightarrow{PA_1}+\overrightarrow{PA_2}\cdot\overrightarrow{PA_2}+\cdots+\overrightarrow{PA_n}\cdot\overrightarrow{PA_n}=\text{常数}$.

\end{document}