\documentclass[UTF8]{ctexart}

\title{\textbf{浙江大学 $2017-2018$ 学年秋冬学期}}
\author{求是数学班《几何学》测验 III}
\date{2018.1}

\usepackage{geometry}
\usepackage{amsmath}
\usepackage{amsfonts}
\usepackage{amssymb}
\usepackage{array}
\usepackage{graphicx}
\usepackage{subfigure}
\usepackage{enumerate}

\linespread{1.75}
\pagestyle{plain}
\geometry{left=2.0cm,top=2.5cm,bottom=2.5cm,right=2.0cm}
\allowdisplaybreaks[4]

\begin{document}
\maketitle
1. (17分) 设直线 $l_1$ 的方程为
\[
\begin{cases}
x-y+z-1=0 \\
y+z=0
\end{cases},
\]
直线 $l_2$ 过点 $M(0,0,-1)$ 且平行于 $\overrightarrow{u}=(2,1,-2)$, 平面 $\pi$ 的方程为 $x+y+z=0$. 求由全体与 $l_1,l_2$ 相交, 且平行于平面 $\pi$ 的直线所构成的曲面的方程.
\\

2. (16分) 求二次曲面 $2x^2+y^2-z^2+3xy+xz-6z=0$ 的标准方程, 并指出这是何种曲面.
\\

3. (17分) 已知椭圆抛物面的方程为 $x^2+2y^2=8z$. 试求所有过点 $P(1,0,3)$, 且与椭圆抛物面的截线是圆的平面的方程.

\end{document}