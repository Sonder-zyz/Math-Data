\documentclass[UTF8]{ctexart}

\title{\textbf{浙江大学 $2019-2020$ 学年秋冬学期}}
\author{求是数学班《常微分方程》测验 I}
\date{2019.10}

\usepackage{geometry}
\usepackage{amsmath}
\usepackage{amsfonts}
\usepackage{amssymb}
\usepackage{array}
\usepackage{graphicx}
\usepackage{subfigure}
\usepackage{enumerate}

\linespread{1.75}
\pagestyle{plain}
\geometry{left=2.0cm,top=2.5cm,bottom=2.5cm,right=2.0cm}
\allowdisplaybreaks[4]

\begin{document}
\maketitle

1. 求下列方程的通解或特解:
\\

(1) $\displaystyle\left(\cos x+\frac{1}{y}\right)\text{d}x+\left(\frac{y-x}{y^2}\right)\text{d}y=0$;
\\

(2) $\displaystyle x\frac{\text{d}y}{\text{d}x}+2y=\sin x, y\left(\frac{\pi}{2}\right)=\frac{1}{\pi}$.
\\

2. 设函数 $f(x)$ 具有一阶连续导数, 且满足:
\[
f(x)=1+x^2+\int_{0}^{x}(x^2-t^2)f^{\prime}(t)\text{d}t.
\]
试求出 $f(x)$.
\\

3. 设 $f(x),g(x),y(x)$ 是 $[x_0,x_1]$ 上的非负连续函数, 求证: 在 $x\in[x_0,x_1]$ 时, 若
\[
y(x)\leq g(x)+\int_{x_0}^{x}f(\tau)y(\tau){\rm d}\tau,
\]
则
\[
y(x)\leq g(x)+\int_{x_0}^{x}f(\tau)g(\tau)e^{\int_{\tau}^{x}f(s){\rm d}s}{\rm d}\tau.
\]
\\

4. 设 $f(x,y)$ 在区域 $D=\left\{(x,y):|x-x_0|\leq a,|y-y_0|\leq b\right\}$ 上连续且存在 $M>0$ 使得 $|f(x,y)|\leq M$. 记 $\displaystyle h=\min\left\{a,\frac{b}{M}\right\}$. 对积分方程 $\displaystyle y(x)=y_0+\int_{x_0}^{x}f(x,y(x)){\rm d}x$ 在区间 $I=[x_0,x_0+h]$ 上构造 Tonelli (托内利) 序列 $y_n(x)$如下:

对任意给定的正整数 $n$, 令 $\displaystyle d_n=\frac{h}{n},x_k=x_0+kd_n(k=0,1,\cdots,n)$, 则分点 $\displaystyle\left\{x_k\right\}_{k=1}^{n}$ 把区间 $I$ 等分成 $n$ 份. 再递推地定义如下函数:
\[
y_n(x)=
\begin{cases}
  y_0, & x\in[x_0,x_1]\\
  y_0+\int_{x_0}^{x-d}f(x,y_n(x)){\rm d}x, & x\in[x_{k-1},x_k](k=2,3,\cdots,n)
\end{cases}
\]

试利用 Ascoli-Arzel\`{a} 定理和 Tonelli 序列来证明初值问题
\[
\frac{{\rm d}y}{{\rm d}x}=f(x,y),y(x_0)=y_0
\]
的解的 Peano 存在性定理.
\\

5. 设在条形区域 $E=\left\{(x,y):0\leq x\leq1,-\infty<y<+\infty\right\}$ 上的函数 $f(x,y)$ 定义如下:
\[
f(x,y)=
\begin{cases}
  0, & x=0,-\infty<y<+\infty \\
  2x, & 0<x\leq1,-\infty\leq y\leq0 \\
  2x-\displaystyle\frac{4y}{x}, &0<x\leq1,0\leq y<x^2 \\
  -2x, & 0<x\leq1,x^2\leq y<+\infty
\end{cases}
\]

(1) 证明 $f(x,y)$ 在 $E$ 内对 $y$ 不满足 Lipschitz 条件, 并计算初值问题 $y^{\prime}=f(x,y),y(0)=0$ 的 Picard 迭代序列;

(2) 验证 $\displaystyle y=\frac{1}{3}x^2$ 是初值问题 $y^{\prime}=f(x,y),y(0)=0$ 的解, 并证明该初值问题的解是唯一的.
\\

6. 试证明初值问题
\[
\frac{{\rm d}y}{{\rm d}x}=(1-y^2)e^{xy^2},y(1)=y_0>0
\]
的解在区间 $[1,+\infty)$ 上存在, 并且当 $x\rightarrow+\infty$ 时积分曲线趋近于直线 $y=1$.

\end{document} 