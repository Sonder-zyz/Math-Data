\documentclass[UTF8]{ctexart}
\title{
\textbf{浙江大学 }20 \underline{ 19 } — 20 \underline{ 20 } \textbf{秋冬学期} \\
\textbf{《常微分方程》课程期末考试试卷}
}
\author{
课程号: \underline{ \quad061Q0056\quad }, 开课学院: \underline{ \quad数学科学学院\quad } \\
考试试卷: \checkmark A 卷、 B 卷 (请在选定项上打\checkmark) \\
考试形式: \checkmark 闭、开卷 (请在选定项上打\checkmark), 允许带\underline{ \quad无\quad  }进场 \\
考试时间: \underline{ \quad2020\quad } 年 \underline{ \quad01\quad } 月 \underline{ \quad09\quad } 日, 考试时间: \underline{ \quad120\quad }分钟 \\
\textbf{诚信考试, 沉着应考, 杜绝违纪}
}
\date{}
\usepackage{geometry}
\usepackage{amsmath}
\usepackage{amsfonts}
\usepackage{amssymb}
\usepackage{array}
\usepackage{graphicx}
\usepackage{subfigure}
\usepackage{enumerate}
\pagestyle{plain}
\geometry{left=2.0cm,top=2.5cm,bottom=2.5cm,right=2.0cm}
\linespread{1.75}
\begin{document}
\maketitle
\begin{center}
考生姓名: \underline{\quad\quad\quad\quad\quad\quad\quad\quad\quad\quad}; 学号: \underline{\quad\quad\quad\quad\quad\quad\quad\quad\quad\quad}; 所属院系: \underline{\quad\quad\quad\quad\quad\quad\quad\quad\quad\quad}
\end{center}

\textbf{注: 该试卷为考生回忆版, 部分试题可能会有错误, 请多多谅解! \textit{@Serapay}}
\\

一、 (32分) 求下列方程 (组) 的特解或通解

(1) $\displaystyle(y-2xy^2\ln x){\rm d}x+x{\rm d}y=0$;

(2) $\displaystyle x^2y^{\prime\prime}-x(2-x)y^{\prime}+(2-x)y=x^4$;

(3) $x^2y^{\prime\prime}+xy^{\prime}+y=6\ln x(x>0)$;

(4) $\displaystyle\frac{{\rm d}\textbf{\textit{X}}}{{\rm d}t}=\textbf{\textit{AX}}$, 其中 $\textbf{\textit{A}}=
\left(\begin{array}{ccc}
3 & -1 & 1 \\
2 & 0 & 1 \\
1 & -1 & 2
\end{array}\right),\textbf{\textit{X}}(0)=(0,1,0)^T$.
\\

二、 (15分) 设方程 $\displaystyle\frac{{\rm d}y}{{\rm d}x}=x^2+(1+y)^2$ 的右行解的最大存在区间是 $[0,T)$. 利用方程 $\displaystyle\frac{{\rm d}y}{{\rm d}x}=(1+y)^2$ 的解是上述方程的一个右行下解这一性质来确定 $T$ 的范围.
\\

三、 (15分) 设连续函数 $f(x,y)$ 在 $(x_0,y_0)$ 的一个邻域内关于 $y$ 单调不减, 证明初值问题 $\displaystyle\frac{{\rm d}y}{{\rm d}x}=f(x,y), y(x_0)=y_0$ 的右行解唯一.
\\

四、 (18分)

(1) 判断方程组 $\begin{cases}
	\displaystyle\frac{\text{d}x}{\text{d}t}=y+x^3\\
	\displaystyle\frac{\text{d}y}{\text{d}t}=-x+ay+y^3\\
    \end{cases}$ 零解的稳定性并说明理由; (原方程记不太清了, 但是线性系统的系数应该没错)

(2) 试找出方程组 $\begin{cases}
              \displaystyle\frac{\text{d}x}{\text{d}t}=x+2y\\
              \displaystyle\frac{\text{d}y}{\text{d}t}=5y-2x+x^3
            \end{cases}$ 的所有奇点, 判断类型并画出相图的草图.
\\

五、 (15分) 若已知方程 $y^{\prime\prime\prime}+a_1(t)y^{\prime\prime}+a_2(t)y^{\prime}=q(t)$ 的一个解, 能否求出该方程的所有解? 给出你的结论并证明之.
\\

六、 (15分) 直接证明初值问题 $\displaystyle\frac{\text{d}y}{\text{d}x}=\sin(xy), x(0)=0$ 解的连续依赖性.
\\

附加题 (10分): 给定线性非齐次方程组 $\displaystyle\frac{\text{d}\boldsymbol{X}}{\text{d}t}=\boldsymbol{AX}+\boldsymbol{R}\left( \boldsymbol{X} \right)$, 其中 $\boldsymbol{R}\left( \boldsymbol{X} \right)$ 是 $\boldsymbol{X}$ 的高阶无穷小, 若已知其线性系统 $\displaystyle\frac{\text{d}\boldsymbol{X}} {\text{d}t}=\boldsymbol{AX}$ 的零解是渐进稳定的, 求证非齐次方程的零解是稳定的.
\end{document} 