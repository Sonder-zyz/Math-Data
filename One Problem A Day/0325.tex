\documentclass[UTF8]{ctexart}
\title{\textbf{每日一题(6)}}
\date{2019.03.25}
\usepackage{geometry}
\usepackage{amsmath}
\usepackage{amsfonts}
\geometry{left=2.0cm,top=1.5cm,bottom=2.5cm,right=2.0cm}
\begin{document}
\maketitle
1. 证明或给出反例: 若 $v_1,v_2,\cdots,v_m$ 线性无关,  $w_1,w_2,\cdots,w_m$ 也线性无关, 则 $v_1+w_1,v_2+w_2,\cdots,v_m+w_m$ 线性无关.

2. 已知复数域上的向量 $\alpha_1,\alpha_2,\cdots,\alpha_n$ 线性无关. 设 $\lambda\in\mathbb{C},$ 那么当 $\lambda$ 取何值时, 向量 $\alpha_1-\lambda\alpha_2,\alpha_2-\lambda\alpha_3,\cdots,\alpha_{n-1}-\lambda\alpha_n,\alpha_n-\lambda\alpha_1$ 线性无关?

1. 解: 反例如下: 取 $m=2,$ $v_1=(1,0),v_2=\left(\displaystyle\frac{1}{2},\displaystyle\frac{3}{4}\right),w_1=(0,1),w_2=\left(\displaystyle\frac{1}{2},\displaystyle\frac{1}{4}\right),$ 则此时 $v_1+w_1=v_2+w_2=(1,1),$ 线性相关.

2. 解: 设 $k_1(\alpha_1-\lambda\alpha_2)+k_2(\alpha_2-\lambda\alpha_3)+\cdots+k_n(\alpha_n-\lambda\alpha_1)=0,$ 其中 $k_i \in \mathbb{C},$ 整理得
\[(k_1-\lambda k_n)\alpha_1+(k_2-\lambda k_1)\alpha_2+\cdots+(k_n-\lambda k_{n-1})\alpha_n=0.\]

因为 $\alpha_1-\lambda\alpha_2,\alpha_2-\lambda\alpha_3,\cdots,\alpha_{n-1}-\lambda\alpha_n,\alpha_n-\lambda\alpha_1$ 线性无关, 所以
\[
\begin{cases}
   k_1-\lambda k_n=0, \\
   -\lambda k_1+k_2=0, \\
   \cdots  \\
   -\lambda k_{n-1}+k_n=0.
\end{cases} 
\]

又因为 $\alpha_1,\alpha_2,\cdots,\alpha_n$ 线性无关, 所以上述方程组只有零解, 故
\[\left|
\begin{array}{cccccc}
  1 &   &   &   &   & -\lambda \\
  -\lambda & 1 &   &   &   &   \\
    & -\lambda & 1 &   &   &   \\
    &   &   & \ddots &   &    \\
    &   &   &   & -\lambda & 1
\end{array}\right|=0,\\
\Longleftrightarrow 1-\lambda^n=0\wedge\lambda=0.
\]
所以 $\lambda$ 的值为 $0$ 或 $\omega^k(\omega=\cos\displaystyle\frac{2\pi}{n}+i\sin\displaystyle\frac{2\pi}{n},k=1,2,\cdots,n)$
\end{document} 