\documentclass[UTF8]{ctexart}
\title{\textbf{每日一题(8)}}
\date{2019.03.28}
\usepackage{geometry}
\usepackage{amsmath}
\usepackage{amsfonts}
\geometry{left=2.0cm,top=1.5cm,bottom=2.5cm,right=2.0cm}
\begin{document}
\maketitle
已知方阵 $\textbf{\textit{A}}=(a_{ij})_{n\times n}, r(\textbf{\textit{A}})=1,\lambda=a_{11}+\cdots+a_{nn}.$ 求证: $\textbf{\textit{A}}^2=\lambda\textbf{\textit{A}}.$

证: 因为 $r(\textbf{\textit{A}})=1,$ 所以存在可逆阵 $\textbf{\textit{P}},\textbf{\textit{Q}},$ 使得
\[
\textbf{\textit{A}}=\textbf{\textit{P}}\left(
\begin{array}{cc}
1 & 0 \\
0 &\textbf{\textit{O}}
\end{array}\right)\textbf{\textit{Q}}=\textbf{\textit{P}}\left(\begin{array}{c}
1 \\
0 \\
\vdots \\
0
\end{array}\right)(1,0,\cdots,0)\textbf{\textit{Q}}=\alpha \beta,
\]
其中:
\[
\alpha=\textbf{\textit{P}}\left(\begin{array}{c}
1 \\
0 \\
\vdots
\\0
\end{array}\right)=
\left(\begin{array}{c}
a_1 \\ \vdots \\a_n
\end{array}\right),
\beta=(1,0,\cdots,0) \textbf{\textit{Q}}=(b_1,\cdots,b_n).
\]
于是
\[
\textbf{\textit{A}}=\alpha \beta =\left( \begin{array}{ccc}
                                    a_1 b_1 & \cdots & a_1 b_n \\
                                    \vdots &   & \vdots \\
                                    a_n b_1 & \cdots & a_n b_n
                                  \end{array} \right),
\lambda = a_1 b_1+ \cdots + a_n b_n=\beta \alpha .
\]
所以 $\textbf{\textit{A}}^2=(\alpha \beta)(\alpha \beta)=\alpha(\beta \alpha)\beta=\lambda \alpha \beta =\lambda \textbf{\textit{A}}.$
\end{document} 