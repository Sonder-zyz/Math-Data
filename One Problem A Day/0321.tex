\documentclass[UTF8]{ctexart}
\title{\textbf{每日一题(2)}}
\date{2019.03.21}
\usepackage{geometry}
\usepackage{amsmath}
\geometry{left=2.0cm,top=1.5cm,bottom=2.5cm,right=2.0cm}
\begin{document}
\maketitle
计算积分: \[ (1) I = \int_{0}^{1} \frac{\ln(1+x)}{1+x^2}{\rm d} x .\]
\[ (2) I_n = \int_{0}^{\frac{\pi}{2}} \sin^n x {\rm d}x. \]

解:$(1)$ 令 $x=\tan t,$ 则${\rm d}x=\sec^2t{\rm d}t,$ 于是
\[
I=\int_{0}^{1}\frac{\ln(1+x)}{1+x^2}{\rm d} x = \int_{0}^{\frac{\pi}{4}}\ln(1+\tan t ){\rm d}t
\]
而 \[
\ln[1+\tan(\frac{\pi}{4}-t)]=\ln(1+\frac{1-\tan t}{1+\tan t})=\ln2 - \ln(1+\tan t),
\]
故 $\ln(1+\tan t)-$ $\displaystyle \frac{1}{2}\ln 2$ 是关于区间 $[0,$ $\displaystyle \frac{\pi}{4}]$ 的中点的奇函数,在该区间上的积分为 $0,$ 因此
\[
I=\int_{0}^{\frac{\pi}{4}}[\ln(1+\tan t)-\frac{1}{2}\ln 2]{\rm d}t+\int_{0}^{\frac{\pi}{4}}\frac{1}{2}\ln 2{\rm d}t=\frac{\pi}{8}\ln 2.
\]
$(2)$ 利用分部积分法,
\begin{align*}
  I_n &=\int_{0}^{\frac{\pi}{2}}\sin^{n-1}x{\rm d}(-\cos x) \\
    &=-\sin^{n-1}x\cos x |_{0}^{\frac{\pi}{2}}+\int_{0}^{\frac{\pi}{2}}\cos x{\rm d}(\sin^{n-1}x) \\
    &=(n-1)\int_{0}^{\frac{\pi}{2}}\sin^{n-2}x\cos^2 x{\rm d}x \\
    &=(n-1)I_{n-2}-(n-1)I_{n-1}
\end{align*}
得到递推公式:
\[
I_n=\frac{n-1}{n}I_{n-2}(n \geq 2)
\]
又 $I_0=$ $\displaystyle \frac{\pi}{2},I_1=1,$ 于是
\[I_n=
\begin{cases}
  \displaystyle \frac{(n-1)!!}{n!!}, & \mbox{$n$ 为奇数},  \\
  \displaystyle \frac{\pi}{2} \cdot \frac{(n-1)!!}{n!!}, & \mbox{$n$ 为偶数}.
\end{cases}
\]
\end{document}
