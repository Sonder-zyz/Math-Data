\documentclass[UTF8]{ctexart}

\title{
\textbf{浙江大学 }20 \underline{ 19 } — 20 \underline{ 20 } \textbf{秋冬学期} \\
\textbf{《抽象代数》课程期末考试试卷}
}

\author{
课程号: \underline{ \quad 061Q0056 \quad }, 开课学院: \underline{ \quad 数学科学学院 \quad } \\
考试试卷: \checkmark A 卷、 B 卷 (请在选定项上打\checkmark) \\
考试形式: \checkmark 闭、开卷 (请在选定项上打\checkmark), 允许带\underline{ \quad 无 \quad  }进场 \\
考试日期: \underline{ \quad 2020 \quad } 年 \underline{ \quad 01 \quad } 月 \underline{ \quad 17 \quad } 日, 考试时间: \underline{ \quad 120 \quad }分钟 \\
\textbf{诚信考试, 沉着应考, 杜绝违纪}
}

\date{}

\usepackage{geometry}
\usepackage{amsmath}
\usepackage{amsfonts}
\usepackage{amssymb}
\usepackage{array}
\usepackage{graphicx}
\usepackage{subfigure}
\usepackage{enumerate}
\pagestyle{plain}
\geometry{top=2.5cm}
\linespread{1.75}

\begin{document}
\maketitle

\begin{center}
考生姓名: \underline{\quad\quad\quad\quad\quad\quad\quad\quad\quad}  学号: \underline{\quad\quad\quad\quad\quad\quad\quad\quad\quad}  所属院系: \underline{\quad\quad\quad\quad\quad\quad\quad\quad\quad}
\end{center}

\textbf{注: 该试卷为考生回忆版, 部分试题可能会有错误, 请多多谅解! \textit{@YZ Zhang}}
\\

1. (15分) 设 $R$ 是一个欧几里得整环, 并有尺度函数 $\sigma:R^*\rightarrow\mathbb{N}$, 满足: 对任意的 $a,b\in R$, 若 $ab\neq0$, 则 $\sigma(a)\le\sigma(ab)$. 其中 $R^*=R\backslash\{0\}$.
\begin{enumerate}[(i)]
  \item 证明: $\varepsilon\in U(R)$ 的充要条件是: $\forall a\in R^*, \sigma(\varepsilon)\le\sigma(a)$, 其中 $U(R)$ 表示 $R$ 的中心;
  \item 若 $a\notin U(R)$, 且对任意的 $b\notin U(R)$ 有 $\sigma(a)\le\sigma(b)$, 证明: $a$ 不可约.\\
\end{enumerate}

2. (15分) 已知域 $F$ 的特征 $\text{ch}(F)=p$, 其中 $p$ 是素数, 定义映射 $\varphi:F\rightarrow F, \varphi(a)=a^p$.
\begin{enumerate}[(i)]
  \item 证明: $\varphi\in\text{End}(F)$;
  \item 证明: $\varphi\in\text{Aut}(F)$ 的充要条件是: $\forall b\in F$, 方程 $x^p-b=0$ 在 $F$ 上有解;
  \item 证明: $F$ 是有限域的充要条件是: $F$ 是 $\mathbb{F}_p$ 上的有限维线性空间; 并进而找出该线性空间的维数与 $F$ 的阶数的关系.\\
\end{enumerate}

3. (10分) 证明: 有单位元的交换环是  IBN 环.
\\

4. (15分) 设 $F$ 是一个域, 记 \[\displaystyle GL_n(F)=\left\{A:A\in M_n(F), A\text{可逆}\right\}, SL_n(F)=
\left\{A:A\in M_n(F), \text{det}(A)=1_F\right\}.\]
\begin{enumerate}[(i)]
  \item 证明: $SL_n(F)$ 是 $GL_n(F)$ 的正规子群, 且 $GL_n(F)/SL_n(F)\cong F\backslash\{0\}$;
  \item 如果把上一问中的域 $F$ 换成带有单位元 $1_R$ 的交换环 $R$, 类似的命题还成立吗? 并证明你的结论.\\
\end{enumerate}

5. (15分) 设有理数域 $\mathbb{Q}$ 的扩张 $K=\mathbb{Q}(\sqrt{3},\sqrt{5},\sqrt{7})$. 计算 $[K:\mathbb{Q}]$, 并证明这是一个 Galois 扩张, 求出 Galois 群 $\text{Aut}_{\mathbb{Q}}K$.
\\

6. (10分) 判断以下数是否可构, 并说明理由: (i) $\sqrt[4]{5+2\sqrt{6}}$; (ii) $\displaystyle\frac{2}{1+\sqrt{7}}$; (iii) $1-\sqrt[5]{27}$.
\\

7. (10分) 分别在 $\mathbb{Q}[x]$ 和 $\mathbb{F}_5[x]$ 上对多项式 $f(x)=2x^3-4x^2+3x-1$ 作因式分解.
\\

8. (10分) 设 $G$ 是一个有限群, $p$ 是素数, $H$ 是 $G$ 的正规子群, 满足 $p\nmid[G:H]$. 证明: 对 $G$ 的任意一个 Sylow $p$-子群 $P$, 有 $P\subset H$.
\\

(\textbf{注:} 最后李方老师捞了大家一把, 出分之前告诉大家, 试卷的满分改成了 $100+10$, 第 $6$ 题是附加题的 $10$ 分)
\end{document} 