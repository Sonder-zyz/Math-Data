\documentclass[UTF8]{ctexart}

\title{\LARGE \textbf{浙江大学 $2018-2019$ 学年春夏学期}}
\author{求是数学班《数学分析 II》 期中考试}
\date{2019.04.28}

\usepackage{geometry}
\usepackage{amsmath}
\usepackage{amsfonts}
\usepackage{amssymb}
\usepackage{array}
\usepackage{graphicx}
\usepackage{subfigure}
\usepackage{enumerate}

\linespread{1.75}
\pagestyle{plain}
\geometry{left=2.0cm,top=2.5cm,bottom=2.5cm,right=2.0cm}
\allowdisplaybreaks[4]
\begin{document}
\maketitle

一、求极限.
\\

(1) $\displaystyle\lim\limits_{n\rightarrow\infty}\frac{1}{n}\left(\sin\frac{\pi}{n}+\cdots+\sin\frac{n-1}{n}\pi\right)$;
\\

(2) $\displaystyle\lim\limits_{x\rightarrow+\infty}\frac{x\int_{0}^{x}e^{t^2}{\rm d}t}{e^{x^2}}$.
\\

二、求积分.
\\

(1) $\displaystyle\int_{-1}^{1}\frac{x+1}{1+\sqrt[3]{x^2}}{\rm d}x$;
\\

(2) $\displaystyle\int_{0}^{\pi}e^{x^2\sin^2x}(\sin x+x\cos x){\rm d}x$;
\\

(3) $\displaystyle\int_{0}^{1}\frac{{\rm d}x}{(2-x)\sqrt{1-x}}$;
\\

(4) $\displaystyle\int_{1}^{+\infty}\frac{\arctan x}{x^2}{\rm d}x$.
\\

三、设曲线
\[
\begin{cases}
  x=a(t-\sin t)\\
  y=a(1-\cos t)
\end{cases},t\in[0,2\pi].
\]

求: (1) 曲线长度; (2)曲线绕 $x$ 轴旋转一周的侧面积.
\\

四、判断下面级数的敛散性, 并说明理由.
\\

(1) $\displaystyle\sum\limits_{n=1}^{\infty}(-1)^n\left(\frac{1}{n^p}-\sin\frac{1}{n^p}\right), p>0$;
\\

(2) $\displaystyle\sum\limits_{n=1}^{\infty}(2-x)(2-x^2)\cdot\cdots\cdot(2-x^{\frac{1}{n}}), x>0$.
\\

五、判断下面广义积分的敛散性, 并说明理由.
\\

(1) $\displaystyle\int_{1}^{+\infty}\frac{\sin x\cos \displaystyle\frac{1}{x}}{x}{\rm d}x$;
\\

(2) $\displaystyle\int_{0}^{1}\frac{\ln x}{(1-x)^2}{\rm d}x$.
\\

六、叙述并证明级数收敛的积分判别法.
\\

七、设 $f\in R([0,1])$ , 证明: $\displaystyle\lim\limits_{n\rightarrow\infty}\int_{0}^{1}f(x)\cos (x^n){\rm d}x=\int_{0}^{1}f(x){\rm d}x$.
\\

八、设数列 ${a_n}$ 单调递减且 $\lim\limits_{n\rightarrow\infty}a_n=0$, 记 $b_n=\displaystyle \min \left\{ a_n,\frac{1}{n} \right\} $. 若 $\displaystyle\sum\limits_{n=1}^{\infty}a_n$ 发散, 证明: $\displaystyle\sum\limits_{n=1}^{\infty}b_n$ 也发散.
\\

九、证明: $\displaystyle\sum\limits_{n=1}^{\infty}\sin(\pi e n!)$ 条件收敛.
\end{document} 