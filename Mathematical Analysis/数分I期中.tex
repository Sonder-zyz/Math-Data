\documentclass[UTF8]{ctexart}

\title{\LARGE \textbf{浙江大学 $2018-2019$ 学年秋冬学期}}
\author{求是数学班《数学分析 I》 期中考试}
\date{2018.11}

\usepackage{geometry}
\usepackage{amsmath}
\usepackage{amsfonts}
\usepackage{amssymb}
\usepackage{array}
\usepackage{graphicx}
\usepackage{subfigure}
\usepackage{enumerate}

\linespread{1.75}
\pagestyle{plain}
\geometry{left=2.0cm,top=2.5cm,bottom=2.5cm,right=2.0cm}
\allowdisplaybreaks[4]

\begin{document}
\maketitle

一、(40分) 计算下面极限:
\\

(1) $\displaystyle\lim\limits_{n\to\infty}\left(1+\frac{1}{2}+\cdots+\frac{1}{n}\right)^\frac{1}{n}$;
\\

(2) $\displaystyle\lim\limits_{n\to\infty}\left(1+\sin\frac{1}{n^2}\right)^n$;
\\

(3) $\displaystyle\lim\limits_{n\to\infty}\frac{\sum{k=1}^{n}\frac{1}{k^{\alpha}}}{n^{1-\alpha}},0<\alpha<1$;
\\

(4) $\displaystyle\underset{n\to\infty}{\underline{\lim}}\left(1+\frac{(-1)^{n-1}}{n} \right)^n$;
\\

(5) $\displaystyle\lim\limits_{x\to+\infty}\frac{x^k}{a^x},a>1,k\in\mathbb{Z}^+$;
\\

(6) $\displaystyle\lim\limits_{x\to0}\frac{\left(a^x-1\right)^2}{1-\cos x},a>0,a\neq1$;
\\

(7) $\displaystyle\lim\limits_{x\to0}\left(\cos x\right)^{\frac{1}{x^2}}$;
\\

(8) $\displaystyle\lim\limits_{x\to a}\left(\frac{\sin x}{\sin a}\right)^{\frac{1}{x-a}},\sin a\neq0$.
\\

二、(10分) 确定函数 $f(x)=\begin{cases}
\displaystyle[x]\sin\frac{1}{x},&x\neq0\\
1, &x=0
\end{cases}$
的不连续点及其类型.
\\

三、(15分) 叙述$\displaystyle\lim\limits_{x\to x_0}f(x)$ 存在的归结原理并证明之.
\\

四、(10分) 设函数 $f(x)$ 在 $[a,b]$ 上无界, 证明: 存在实数 $\xi\in[a,b]$ 及点列 $\{x_n\}\subset [a,b]$, 使得 $\displaystyle\lim\limits_{n\to\infty}x_n=\xi,\lim\limits_{n\to\infty}f(x_n)=\infty$.
\\

五、(15分) 证明:

(1) 当 $\alpha\in[-1,0]$时, $f(x)=x^{\alpha}\sin x$ 在 $(0,+\infty)$ 一致连续;

(2) 当 $\alpha\in(-\infty,-1)\cup(0,+\infty)$ 时, $f(x)=x^{\alpha}\sin x$ 在 $(0,+\infty)$ 不一致连续.
\\

六、(10分) 设 $\displaystyle a>0,0<x_1<a^{-\frac{1}{4}},x_{n+1}=x_n-ax_n^5(n\geq1)$.

(1) 证明: $\lim\limits_{n\to\infty}x_n=0$;

(2) 求极限 $\lim\limits_{n\to\infty}\sqrt[4]{n}x_n$.


\end{document}