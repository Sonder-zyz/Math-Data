\documentclass[UTF8]{ctexart}
\title{\LARGE \textbf{浙江大学 $2018-2019$ 学年春夏学期}}
\author{求是数学班《高等代数 II》测验 I}
\date{2019.03.22}
\usepackage{geometry}
\usepackage{amsmath}
\usepackage{amsfonts}
\geometry{left=2.0cm,top=2.5cm,bottom=2.5cm,right=2.0cm}
\linespread{1.75}
\pagestyle{plain}
\begin{document}
\maketitle
1. 设矩阵$\textbf{\textit{A}}=\left(\begin{array}{ccc}
                             a & -1 & c \\
                             5 & b & 3 \\
                             1-c & 0 & -a
                           \end{array}\right),$ ${\rm det}(\textbf{\textit{A}})=-1,$
且 $\textbf{\textit{A}}$ 的转置 $\textbf{\textit{A}}^T$ 有一个特征值 $\lambda_0,$ 属于 $\lambda_0$ 的一个特征向量为 $(-1,-1,1)^T.$ 求 $a,b,c,\lambda_0$ 的值.
\\

2. 设 $\textbf{\textit{A}}\in M_n(\mathbb{F})$ 有两个不同的特征值 $\lambda_1,\lambda_2,$ 且 ${\rm dim}(E_{\lambda_1})=n-1.$ 证明: $\textbf{\textit{A}}$ 可对角化.
\\

3. 设 $V$ 是次数 $\leq 2$ 的实多项式全体构成的线性空间, $T\in L(V)$ 定义如下:
\[ T\left(f(x)\right)=f(x)+\lambda f'(x).\]
求 $T$ 的特征值, 并且对每个特征值, 求属于它的特征子空间.
\\

4. 设  $\textbf{\textit{A}},$ $\textbf{\textit{B}},$  $\textbf{\textit{C}}\in M_n(\mathbb{F}),$  $\textbf{\textit{A}},$  $\textbf{\textit{B}}$ 各有 $n$ 个不同的特征值. $f(t)$ 是  $\textbf{\textit{A}}$ 的特征多项式, 且 $f(\textbf{\textit{B}})$ 是可逆阵, 求证: 矩阵$ \left(\begin{array} {cc}
\textbf{\textit{A}}  &  \textbf{\textit{C}}  \\
\textbf{\textit{O}}  &  \textbf{\textit{B}}
\end{array}\right)$ 可对角化.
\\

5. 设 $\sigma :V\longrightarrow V$ 是有限维线性空间 $V$ 上的一个同构映射. 记 $E(\sigma , \lambda)$ 为 $\sigma$ 的属于特征值 $\lambda$ 的特征子空间.

(1)若 $\lambda$ 是 $\sigma$ 的特征值, 证明: $\lambda \neq 0;$

(2)若 $\lambda$ 是 $\sigma$ 的特征值, 证明: $E(\sigma , \lambda)=E(\sigma ^{-1}, \lambda ^{-1}).$
\\

6. 设 $\textbf{\textit{X}},\textbf{\textit{Y}}$ 分别是 $n \times m$ 矩阵和 $m \times n$ 矩阵, 且满足 $\textbf{\textit{YX}}=\textbf{\textit{I}}_m.$ 令 $\textbf{\textit{A}}=\textbf{\textit{I}}_n+\textbf{\textit{XY}},$ 证明: $\textbf{\textit{A}}$ 可对角化.
\\

7. 设 $\textbf{\textit{A}} \in M_n(\mathbb{R}), \textbf{\textit{A}}=\left( A_{ij}\right).$ 若对于任意的 $i=1,2,\cdots,n$ 有 $A_{ii} > \sum \limits_{j\neq i}\left|A_{ij}\right|,$ 则称 $\textbf{\textit{A}}$ 是“严格对角占优阵”. 证明: 严格对角占优阵的特征值不为零.
\\

8. 设 $\textbf{\textit{A}},\textbf{\textit{B}} \in M_n(\mathbb{C}),$ 当他们可交换时, 他们可以被同时相似化(即: 存在可逆阵 $\textbf{\textit{P}}$, 使得 $\textbf{\textit{PAP}}^{-1}$ 和 $\textbf{\textit{PBP}}^{-1}$ 同时为上三角阵) 若将“可交换”这一条件改为: 存在非零常数 $c,$ 使得 $\textbf{\textit{AB}}=c\textbf{\textit{BA}},$ 类似的结论是否成立? 说明你的理由.
\end{document} 