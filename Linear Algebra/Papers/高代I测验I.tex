\documentclass[UTF8]{ctexart}
\title{\LARGE \textbf{浙江大学 $2018-2019$ 学年秋冬学期}}
\author{求是数学班《高等代数 I》测验 I}
\date{2018.10}
\usepackage{geometry}
\usepackage{amsmath}
\usepackage{amsfonts}
\usepackage{amssymb}
\usepackage{array}
\usepackage{graphicx}
\usepackage{subfigure}
\usepackage{enumerate}
\pagestyle{plain}
\geometry{left=2.0cm,top=2.5cm,bottom=2.5cm,right=2.0cm}
\linespread{1.75}
\begin{document}
\maketitle
1. 设 $A=\left(\begin{array}{ccc}
          1 & 2 & -2 \\
          2 & 1 & 2 \\
          3 & 0 & -4
        \end{array}\right)$, $\alpha=\left(\begin{array}{c}
                                             x \\
                                             1 \\
                                             1
                                           \end{array}\right)$, 若 $A\alpha$ 与 $\alpha$ 线性相关, 求 $x$ 的值.
\\

2. 设 $D$ 为 $\mathbb{R}_n[x]$ 上的 线性映射, 对 $\mathbb{R}_n[x]$ 中的多项式 $p(x)=a_0+a_1x+\cdots+a_nx^n$, $D(p(x)):=a_1+2a_2x+\cdots+na_nx^{n-1}$.

(1) 求 $D$ 关于基 $\beta={1,x,\cdots,x^n}$ 的矩阵;

(2) 已知 $\gamma=\{1,k_1x,\cdots,k_nx^n\}$ 是 $\mathbb{R}_n[x]$ 的另一组基, $D$ 关于 $\gamma$ 的矩阵为
\[
\left(\begin{array}{cccccc}
  0 & 1 & 0 & \cdots & 0 & 0 \\
  0 & 0 & 1 & \cdots & 0 & 0 \\
  \cdots & \cdots & \cdots & \cdots & \cdots & \cdots \\
  0 & 0 & 0 & \cdots & 1 & 0 \\
  0 & 0 & 0 & \cdots & 0 & 1 \\
  0 & 0 & 0 & \cdots & 0 & 0
\end{array}\right),
\]
求 $k_1,k_2,\cdots,k_n$ 的值.
\\

3. 设 $V$ 是由如下四个矩阵生成的 $M_2(\mathbb{F})$ 的子空间:
\[
A_1=\left(\begin{array}{cc}
      -1 & 4 \\
      2 & 0
    \end{array}\right),
A_2=\left(\begin{array}{cc}
      5 & 1 \\
      0 & 3
    \end{array}\right),
A_3=\left(\begin{array}{cc}
      3 & -2 \\
      -1 & 4
    \end{array}\right),
A_4=\left(\begin{array}{cc}
      -2 & 9 \\
      4 & -5
    \end{array}\right).
\]

(1) 求 $\text{dim }V$ 和 $V$ 的一组基;

(2) 映射 $f : V \rightarrow F$ 定义为: $f(A)=\text{tr }A$, (其中 $\text{tr }A$ 表示矩阵 $A$ 的迹), $\text{ker}f=\left\{A\in V \left| f(A)=0\right\}\right.$, 求 $\text{dim ker }f$ 并找出 $\text{ker }f$ 的一组基.
\\

4. 设 $T$ 是 $n$ 维线性空间 $V$ 上的一个线性变换, 证明: 可以在 $V$ 中选取这样的两组基 $\alpha_1,\alpha_2,\cdots,\alpha_n$ 和 $\beta_1,\beta_2,\cdots,\beta_n$, 使得对 $V$ 中的任意向量 $v$, 若 $v=\displaystyle\sum_{i=1}^{n}k_i\alpha_i$, 则 $T(v)=\displaystyle\sum_{i=1}^{r}k_i\beta_i(1\leq r\leq n)$.
\\

5. 设 $V$ 是数域 $\mathbb{F}$ 上的一个 $n$ 维线性空间, $\alpha_1,\alpha_2,\cdots,\alpha_n$ 是它的一组基, 记$V_1=\text{span}\{\alpha_1+\alpha_2+\cdots+\alpha_n\}$,  $V_2=\displaystyle\left\{\sum_{i=1}^{n}k_i\alpha_i \left| \sum_{i=1}^{n}k_i=0, k_i\in\mathbb{F}\right\}\right.$.

(1) 证明: $V_2$ 是 $V$ 的子空间;

(2) 证明: $V=V_1\oplus V_2$.
\\

6. 设 $V_1,V_1,\cdots,V_n$ 和 $W$ 都是线性空间 $V$ 的子空间, 证明: 如果 $W$ 包含在 $\displaystyle\bigcup_{i=1}^{n}V_i$ 中, 那么 $W$ 必包含在某个 $V_i(1\leq i\leq n)$ 中.
\\

7. 向量组 $\alpha_1,\alpha_2,\cdots,\alpha_s$ 线性无关, 且可以由向量组 $\beta_1,\beta_2,\cdots,\beta_t$ 线性表出. 证明: 必存在某个向量 $\beta_j(1\leq j\leq t)$, 使得向量组 $\beta_j,\alpha_2,\cdots,\alpha_s$ 线性无关.
\\

8. 设 $V$ 是实数域 $\mathbb{R}$ 上的一个线性空间, 定义 $V_{\mathbb{C}}:=\left\{u+\text{i}v\left|u,v\in V\right\}\right.$. 试给出 $V_{\mathbb{C}}$ 上的加法运算和在复数域 $\mathbb{C}$ 上的数乘运算, 使得 $V_{\mathbb{C}}$ 成为复数域 $\mathbb{C}$ 上的线性空间. 并试着猜测 $V$ 的基和维数与 $V_{\mathbb{C}}$ 的基和维数之间的关系.
\end{document} 