\documentclass[UTF8]{ctexart}
\title{\LARGE \textbf{浙江大学 $2018-2019$ 学年春夏学期}}
\author{求是数学班《高等代数 II》测验 III}
\date{2019.05.31}
\usepackage{geometry}
\usepackage{amsmath}
\usepackage{amsfonts}
\geometry{left=2.0cm,top=2.5cm,bottom=2.5cm,right=2.0cm}
\linespread{1.75}
\pagestyle{plain}
\begin{document}
\maketitle
1. 求下面线性方程组的 ``解" :
\[
\begin{cases}
  x_1+x_2+x_3+x_4=2 \\
  x_1-2x_3+x_4=-1 \\
  x_1-x_2+x_3+x_4=2
\end{cases}
\]
\\

2. 设 $A$ 是正定阵, $A=U\Sigma V^*$ 是 $A$ 的奇异值分解, 证明: $U=V$.
\\

3. 设 $A$ 是 $n$ 阶正定阵, 向量组 $\beta_1,\cdots,\beta_s$ 满足 $\beta_i^TA\beta_j=0(1\leq i<j\leq s)$. 试问向量组 $\beta_1,\cdots,\beta_s$ 的秩可能是多少? 证明你的结论.
\\

4. 设 $A,B$ 为同阶正定阵, 若 $A>B$ (即 $A-B$ 是正定阵), 试问是否一定有 $A^2>B^2$? 为什么?
\\

5. 设实二次型 $f(X)=X^TAX$, $\lambda$ 是 $A$ 的特征值. 证明: 存在非零向量 $\alpha=(k_1,\cdots,k_n)^T$, 使得 $f(\alpha)=\lambda(k_1^2+\cdots+k_n^2)$.
\\

6. 设 $f$ 是双线性型, 且对任意的 $x,y,z$ 有
\[
f(x,y)f(z,x)=f(y,x)f(x,z).
\]
证明: $f$ 是对称的或者反对称的.
\\

7. 设 $M_{2r+1}(\mathbb{F})$ 是数域 $\mathbb{F}$ 上的全体 $2r+1$ 阶方阵组成的集合, 记
\[
M=\left(
\begin{array}{ccc}
  2 & O & O \\
  O & O & I_r \\
  O & I_r & O
\end{array}
\right)
\]
是分块矩阵, 其中 $I_r$ 是 $r$ 阶单位阵. 设
\[
W=\left\{X\in M_{2r+1}(\mathbb{F}) \left| X^TM+MX=O \right\}\right..
\]
对 $X\in W$, 设 $e^X=\displaystyle\sum\limits_{k=0}^{\infty}\frac{X^k}{k!}$. 已知: $e^X\in M_{2r+1}(\mathbb{F})$.

(1)求 dim $W$ 和 $W$ 的一组基;

(2)证明: 对任意的 $X\in W$ 都有 det $(e^X)=1$;

(3)设列向量空间 $\mathbb{F}^{2r+1}$ 上的一个双线性型 $f$ 关于标准基 $\{e_1,\cdots,e_{2r+1}\}$ 的矩阵表示为上述 $M$, 证明: 对任意的 $X\in W$ 和列向量 $\alpha,\beta\in\mathbb{F}^{2r+1}$, 都有
\[
f(e^X\alpha,e^X\beta)=f(\alpha,\beta).
\]
\end{document} 